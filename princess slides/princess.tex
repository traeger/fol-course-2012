% !TEX TS-program = latex
% !TEX encoding = UTF-8 Unicode

\documentclass{beamer} % Standardklasse für Beamer
\beamertemplatenavigationsymbolsempty % kleinen Navigationsleiste am unteren Rand
\setbeamertemplate{footline}[frame number] % Seitenzahlen in Fußzeile einfügen

\usepackage[utf8]{inputenc}
\usepackage[english,ngerman]{babel}

\usepackage{amsfonts}
\usepackage{mathptmx}
\usepackage[scaled=0.92]{helvet}
\usepackage{courier}
\usepackage[T1]{fontenc}

\usepackage{listings}

%%%%%%%%
% listings
%%%%%%%%
\lstset{ %
  basicstyle=\tiny
}

\usepackage{environ}

\newenvironment{page}[1]
	{\begin{frame}[containsverbatim]{#1}}
    	{\end{frame}}

\title{Princess}
\subtitle{Theorem Proving in First-Order Logic modulo Linear Integer Arithmetic}

\begin{document}

\selectlanguage{english}
\maketitle

\begin{page}{Authors}
\begin{itemize}
	\item Philipp Rümmer (main developer) (Department of Information Technology, Uppsala University)
	\item Angelo Brillout (Eidgenössische Technische Hochschule Zürich)
\end{itemize}

\end{page}
%%
%%
%%
\begin{page}{Facts}
\begin{itemize}
	\item uses Presburger arithmetic
	\item can eliminate quantifiers
	\item handling of (partial and total) functions via an encoding into uninterpreted predicates
	\item input formates
	\begin{itemize}
		\item native format
		\item SMT-LIB 2 
		\item FOF/TFF/CNF dialects of TPTP.
	\end{itemize}
	\item handling of the theory of arrays via user-specified axioms
\end{itemize}

\end{page}
%%
%%
%%
\begin{page}{Native Input Language}

\begin{lstlisting}
\exists int grandson_years, son_years, my_years,
            grandson_months, son_weeks, grandson_days; (
  grandson_days = grandson_months * 4 * 7 &
  grandson_months = grandson_years * 12 &
  son_weeks = son_years * 12 * 4 &
  grandson_days = son_weeks &
  grandson_months = my_years &
  grandson_years + son_years + my_years = 140
)
\end{lstlisting}
\begin{lstlisting}
Formula is valid, satisfying assignment for the existential constants is:
true
\end{lstlisting}

\end{page}
%%
%%
%%
\begin{page}{Native Input Language}

\begin{lstlisting}
\existentialConstants {
  int my_years;
}

\problem {
  \exists int grandson_years, son_years,
              grandson_months, son_weeks, grandson_days; (
    grandson_days = grandson_months * 4 * 7 &
    grandson_months = grandson_years * 12 &
    son_weeks = son_years * 12 * 4 &
    grandson_days = son_weeks &
    grandson_months = my_years &
    grandson_years + son_years + my_years = 140
  )
}
\end{lstlisting}
\begin{lstlisting}
Formula is valid, satisfying assignment for the existential constants is:
(my_years + -84 = 0)
\end{lstlisting}

\end{page}
%%
%%
%%
\begin{page}{Presburger arithmetic}

\begin{itemize}
\item $\lnot(0 = x + 1)$
\item $x + 1 = y + 1 \to x = y$
\item $x + 0 = x$
\item $(x + y) + 1 = x + (y + 1)$
\item $(P(0) \land \forall x(P(x) \to P(x + 1))) \to \forall y P(y).$
\end{itemize}

\end{page}
%%
%%
%%
\begin{page}{TPTP Performance}

\begin{itemize}
	\item TPTP Performance (June 2012)
	\begin{itemize}
		\item TFA: Winner, place 1/3
		\item FOF: place 12/16
		\item FOF@Turing: place 12/16
	\end{itemize}
	\item AUFLIA Benchmarks (November 2011) \\
		benchmarks on containing arrays, uninterpreted functions, integer arithmetic, and quantifiers \\
		place 2
\end{itemize}

\end{page}
%%
%%
%%

\end{document}